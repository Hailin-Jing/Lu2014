\begin{ParaColumn}[\bisection*{$K_0$-Consolidated Condition}{$K_0$固结状态}]
    
    When the state-dependent Mohr-Coulomb hardening model was used to simulate the $K0$-consolidated undrained triaxial tests, the evolution of $M$ needed to be changed as

    \switchcolumn

    当使用状态依赖的摩尔库伦硬化模型来模拟$K_0$固结不排水三轴试验时,$M$的演变需要更改为

    \CrossColumnText{
        \begin{align}
            M=M_{0}+\left(M_{p}-M_{0}\right) \frac{\varepsilon_{s}^{p}}{A+\varepsilon_{s}^{p}}
        \end{align}
    }
    \switchcolumn*

    \noindent
    where $M_0$ = stress ratio after $K_0$ consolidation.

    \switchcolumn

    \noindent
    式中,$M_0$为$K_0$固结后的应力比。

    \switchcolumn*

    The hardening modulus under $K_0$-consolidated condition becomes

    \switchcolumn

    $K_0$固结条件下的硬化模量变为

    \CrossColumnText{
        \begin{align}
            H_{p}=p^{\prime}\left(M_{p}-M_{0}\right) \frac{A}{\left(A+\varepsilon_{e p}\right)^{2}}
        \end{align}
    }
    \switchcolumn*

    It should be noted that the proposed model is different from the standard Mohr-Coulomb hardening models, which treat soils in different initial states as different materials. By using the state parameter to calibrate the influence of void ratio and mean stress on the peak stress ratio and stress dilatancy, the proposed model could unify modeling the constitutive relationship of sands under different initial void ratios, confining stresses, and consolidation conditions. This model takes advantage of the other critical state models \citep{Jefferies1993,Wood1994, Li1999, Wan2004,Andrade2008}, but it is quite simple and could be applied easily in engineering.

    \switchcolumn

    应该注意的是,提出的模型与标准的摩尔库伦硬化模型不同,后者将处于不同初始状态的土体视为不同的材料。 通过使用状态参数来校准孔隙比和平均应力对峰值应力比和应力剪胀比的影响,该模型可以统一建模在不同初始孔隙比,约束应力和固结条件下砂土的本构关系。 该模型利用了其他临界状态模型\citep{Jefferies1993,Wood1994, Li1999, Wan2004,Andrade2008},但是它非常简单,可以很容易地在工程中应用。
\end{ParaColumn}