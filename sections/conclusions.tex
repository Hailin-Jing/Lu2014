\begin{ParaColumn}[\bisection*{Conclusions}{结论}]

    A state-dependent nonassociated Mohr-Coulomb hardening elastoplasticity model and second-order work criterion were proposed to study the static liquefaction under the undrained triaxial condition. The vanishing value of the determinant of the symmetric part of the elastoplastic modulus tensor was used to predict the potentially unstable state of soils, and the sign change in second-order work denoted the onset of static liquefaction. The static liquefaction initiated only when the undrained stress path occurred along with the potentially unstable stress path, or when the soil stayed stable even the stress state was located in the potentially unstable region.

    \switchcolumn

    为了研究在不排水的三轴条件下的静态液化,提出了一种状态相关的非关联摩尔库伦硬化弹塑性模型和二阶功准则。 弹塑性模量张量的对称部分的行列式值用于预测土体的潜在不稳定状态,并且二阶功的符号变化表示静态液化的开始。 静态液化仅在不排水的应力路径与潜在的不稳定应力路径同时发生时发生,或者当土体保持稳定甚至应力状态位于潜在不稳定区域时才开始。

    \switchcolumn*

    The proposed model and instability criteria were used to analyze a series of isotropically consolidated and $K_0$-consolidated undrained triaxial tests. The results showed that static liquefaction is prone to occur in loose sands, although when the sand is dense enough, it would be prevented by the tendency of dilatancy even if the state of sands becomes potentially unstable. The static liquefaction occurred at the hardening regime of sands before the plastic limit failure criterion was reached; its occurrence induced the drop of shear resistance and the increase of pore-water pressure.

    \switchcolumn

    采用所提出的模型和失稳准则进行了一系列的等向固结和$K_0$固结不排水三轴试验。结果表明:松散砂土容易发生静态液化,但当砂土密实度足够大时,即使砂土的状态变得潜在不稳定,其膨胀趋势也会阻止静态液化的发生。静力液化发生在砂体硬化区,未达到塑性极限破坏准则;其发生导致剪切阻力下降,孔隙水压力增大。
\end{ParaColumn}