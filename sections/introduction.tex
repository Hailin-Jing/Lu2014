\begin{ParaColumn}[\bisection*{Introduction}{介绍}]

    The instability of soils that occurs before the peak failure often demonstrates as a localized or diffuse mode. The localized mode of instability appears in dense sand when loaded under drained condition, and it has been studied for a long time \citep{Rudnicki1975,Andrade2006,Lu2011}. For very loose sands, when monotonically loaded under an undrained condition, static liquefaction takes place; this kind of instability is different from the localized mode and keeps its homogeneous deformation. Static liquefaction is a typical and important characteristic of loosely compacted saturated soil, and it has been used in the stability analysis of saturated slope \citep{Lade1992,Ellison2009}.

    \switchcolumn

    在峰值破坏之前发生的土体不稳定性通常表现为局部或扩散模式。 在排水条件下加载时,局部失稳模式出现在稠密的沙子中,对此进行了长期的研究\citep{Rudnicki1975,Andrade2006,Lu2011}。 对于非常疏松的砂,在不排水的情况下单调加载时,会发生静态液化。 这种不稳定性不同于局部模式,并保持其均匀变形。 静态液化是松散压实的饱和土的典型和重要特征,已用于饱和边坡的稳定性分析\citep{Lade1992,Ellison2009}。

    \switchcolumn*

    Static liquefaction has been studied experimentally by triaxial tests \citep{Sladen1985, Lade1990,Yamamuro1997,Doanh1997}, plane-strain tests \citep{Chu2008}, and ring-shear tests \citep{Liu2011}. The influence of K0 consolidation on static liquefaction was also studied experimentally \citep{Fourie2005, Chu2008}. Based on the summary of existing experimental results, the onset of static liquefaction instability could be determined by the instability line \citep{Lade1990} or collapse surface \citep{Sladen1985} in stress space. The experimental results \citep{Daouadji2010,Wanatowski2012} showed the main factors affecting instability are the stress state, drainage condition, loading mode, and material state. By summing up a series of experimental results, the undrained instability has been characterized by critical state theory \citep{Rahman2011,Bedin2012}. To describe the influence of the initial material state on instability, \citet{Yang2002} expressed the instability line as a function of the current material state, and \citet{Rahman2011} defined the instability line as a function of the equivalent granular state parameter.

    \switchcolumn

    静态液化已通过三轴试验\citep{Sladen1985, Lade1990,Yamamuro1997,Doanh1997},平面应变试验\citep{Chu2008}和环刀剪切试验\citep{Liu2011}进行了实验研究。还通过实验研究了K0固结对静态液化的影响\citep{Fourie2005, Chu2008}。根据现有实验结果的总结,静态液化不稳定性的发生可以通过应力空间中的不稳定性线\citep{Lade1990}或坍塌表面\citep{Sladen1985}来确定。实验结果\citep{Daouadji2010,Wanatowski2012}表明,影响失稳的主要因素是应力状态,排水条件,加载模式和材料状态。通过总结一系列实验结果,通过临界状态理论对不排水的不稳定性进行了表征\citep{Rahman2011,Bedin2012}。为了描述初始材料状态对不稳定性的影响,\citet{Yang2002}将不稳定性线表示为当前材料状态的函数,\citet{Rahman2011}将不稳定性线定义为等效颗粒状态参数的函数。

    \switchcolumn*

    Compared with the experimental studies, there is lack of theoretical study on static liquefaction. \citet{Borja2006} adopted the bifurcation theory to study the initiation of liquefaction instability in saturated soils. \citet{Andrade2009} presented a practical mathematical framework for predicting the liquefaction instability, which has been shown to coincide with the loss of uniqueness of material response \citep{Nova1994}. \citet{Buscarnera2011} studied diffuse instability by loss of controllability and derived the mathematical condition under different loading conditions. Based on the Nor-Sand model \citep{Jefferies1993, Andrade2008}, the critical hardening modulus corresponding to the onset of static liquefaction was obtained and adopted in numerical modeling of submarine slope \citep{Ellison2009}. In these works, static liquefaction was predicted to occur at the strain-softening stage; instability can also emerge within stable single-phase solids owing to the interaction between the solid matrix and fluid flow \citep{Bardet2002}.

    \switchcolumn

    与实验研究相比,缺乏关于静态液化的理论研究。 \citet{Borja2006}运用分叉理论研究了饱和土体中液化不稳定性的起因。 \citet{Andrade2009}提出了一个实用的数学框架来预测液化不稳定性,这已经证明与材料响应唯一性的丧失相吻合\citep{Nova1994}。\citet{Buscarnera2011}研究了由于失控引起的扩散不稳定性,并推导了不同载荷条件下的数学条件。基于Nor-Sand模型\citep{Jefferies1993, Andrade2008},获得了对应于静态液化开始的临界硬化模量,并将其用于海底边坡的数值模拟中\citep{Ellison2009}。在这些工作中,预计在应变软化阶段会发生静态液化。由于固体基质和流体之间的相互作用,在稳定的单相固体中也可能出现不稳定性\citep{Bardet2002}。

    \switchcolumn*

    This paper proposes a model to calibrate the stress-strain relationships and study the onset of static liquefaction under different initial material states, consolidation states, and confining conditions. After a view of the critical state theory, a state-dependent Mohr-Coulomb elastoplasticity hardening model is proposed by using material state–dependent strength and stress dilatancy. Based on the second-order work criterion, the theoretical condition for static liquefaction is proposed and used to predict the results of isotropically consolidated and $K_0$-consolidated undrained triaxial compression tests. Finally, the influence of the initial material states of sands at the onset of static liquefaction is discussed.

    \switchcolumn

    本文提出了一个校准应力-应变关系的模型,并研究了在不同初始材料状态,固结状态和约束条件下静态液化的开始。 根据临界状态理论,通过使用材料状态相关的强度和应力膨胀率,提出了状态相关的摩尔库伦弹塑性硬化模型。 基于二阶功准则,提出了静态液化的理论条件,并将其用于各向同性固结和$K_0$固结不排水三轴压缩试验的结果。 最后,讨论了砂土初始材料状态在静态液化开始时的影响。

\end{ParaColumn}